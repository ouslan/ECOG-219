\documentclass{article}

\usepackage{fancyhdr}
\usepackage{extramarks}
\usepackage{amsmath}
\usepackage{amsthm}
\usepackage{amsfonts}
\usepackage{tikz}
\usepackage[plain]{algorithm}
\usepackage{algpseudocode}
\usepackage{pgfplots}
\pgfplotsset{compat=1.18}
\usetikzlibrary{automata,positioning}

%
% Basic Document Settings
%

\topmargin=-0.45in
\evensidemargin=0in
\oddsidemargin=0in
\textwidth=6.5in
\textheight=9.0in
\headsep=0.25in

\linespread{1.1}

\pagestyle{fancy}
\lhead{\hmwkAuthorName}
\chead{\hmwkClass\ (\hmwkClassInstructor): \hmwkTitle}
\rhead{\firstxmark}
\lfoot{\lastxmark}
\cfoot{\thepage}

\renewcommand\headrulewidth{0.4pt}
\renewcommand\footrulewidth{0.4pt}

\setlength\parindent{0pt}

%
% Create Problem Sections
%

\newcommand{\enterProblemHeader}[1]{
	\nobreak\extramarks{}{Problem \arabic{#1} continued on next page\ldots}\nobreak{}
	\nobreak\extramarks{Problem \arabic{#1} (continued)}{Problem \arabic{#1} continued on next page\ldots}\nobreak{}
}

\newcommand{\exitProblemHeader}[1]{
	\nobreak\extramarks{Problem \arabic{#1} (continued)}{Problem \arabic{#1} continued on next page\ldots}\nobreak{}
	\stepcounter{#1}
	\nobreak\extramarks{Problem \arabic{#1}}{}\nobreak{}
}

\setcounter{secnumdepth}{0}
\newcounter{partCounter}
\newcounter{homeworkProblemCounter}
\setcounter{homeworkProblemCounter}{1}
\nobreak\extramarks{Problem \arabic{homeworkProblemCounter}}{}\nobreak{}

%
% Homework Problem Environment
%
% This environment takes an optional argument. When given, it will adjust the
% problem counter. This is useful for when the problems given for your
% assignment aren't sequential. See the last 3 problems of this template for an
% example.
%
\newenvironment{homeworkProblem}[1][-1]{
	\ifnum#1>0
		\setcounter{homeworkProblemCounter}{#1}
	\fi
	\section{Problem \arabic{homeworkProblemCounter}}
	\setcounter{partCounter}{1}
	\enterProblemHeader{homeworkProblemCounter}
}{
	\exitProblemHeader{homeworkProblemCounter}
}

%
% Homework Details
%   - Title
%   - Due date
%   - Class
%   - Section/Time
%   - Instructor
%   - Author
%

\newcommand{\hmwkTitle}{Problem Set\ \#5}
\newcommand{\hmwkDueDate}{Jun 26, 2025}
\newcommand{\hmwkClass}{ECON 219}
\newcommand{\hmwkClassInstructor}{Dr. Sergio Urzua}
\newcommand{\hmwkAuthorName}{\textbf{Alejandro Ouslan}}

%
% Title Page
%

\title{
	\vspace{2in}
	\textmd{\textbf{\hmwkClass:\ \hmwkTitle}}\\
	\normalsize\vspace{0.1in}\small{Due\ on\ \hmwkDueDate}\\
	\vspace{0.1in}\large{\textit{\hmwkClassInstructor}}
	\vspace{3in}
}

\author{\hmwkAuthorName}
\date{}

\renewcommand{\part}[1]{\textbf{\large Part \Alph{partCounter}}\stepcounter{partCounter}\\}

%
% Various Helper Commands
%

% Useful for algorithms
\newcommand{\alg}[1]{\textsc{\bfseries \footnotesize #1}}

% For derivatives
\newcommand{\deriv}[1]{\frac{\mathrm{d}}{\mathrm{d}x} (#1)}

% For partial derivatives
\newcommand{\pderiv}[2]{\frac{\partial}{\partial #1} (#2)}

% Integral dx
\newcommand{\dx}{\mathrm{d}x}

% Alias for the Solution section header
\newcommand{\solution}{\textbf{\large Solution}}

% Probability commands: Expectation, Variance, Covariance, Bias
\newcommand{\E}{\mathrm{E}}
\newcommand{\Var}{\mathrm{Var}}
\newcommand{\Cov}{\mathrm{Cov}}
\newcommand{\Bias}{\mathrm{Bias}}

\begin{document}

\maketitle

\pagebreak

% Problem 1
\begin{homeworkProblem}
	Asume the following smooth production function:
	$$Q= Q(K,L)$$
	with positive marginal productivities. Let $w$ and $r$ the prices of
	labor and capital, respectively.
	\begin{enumerate}
		\item Formulate the problem of minimizing costs subject to the technology.

		\item Explain under what conditions you might have to consider the xase of corner
		      solution (optimal labor or capital equal to zero). Provide an example.
		\item Assuming interior solution present the first order conditions.
		      Provide an economic interpretation to the optimality condition. In you answer,
		      refer to the Langrange multiplier.
		\item Provide a graphical representation of the resulting optimal input combination.
		\item Present the second order condition.
		      Explain how the strict convexity of the isoquants would ensure a minimum cost.
		      Explain how quasi-concave production function can generate everywhere strictly convex,
		      downard-sloping isoquants.
		\item Now, assume $Q=AL^\alpha K^\beta$. Show that the expansion path (optimal combinations
		      of capital and labor for different isocosts) is characterized by a linear combination.
		\item Show the previous result holds for all homogeneous produciton functions.
	\end{enumerate}
\end{homeworkProblem}
\begin{homeworkProblem}
	Consider the following model:
	$$Y=X\beta + \epsilon$$
	where the standard assumption securing OLS delivers BLUE estimators hold. Assume the error terms
	is normally distributed with mean $0$ and variance $\sigma^2$.
	\begin{enumerate}
		\item Present the likeligood function and optimization problem
		\item Present the first and second order conditions.
		\item Generate a sample of $1000$ observations under the following parameterization:
		      $$Y=X\beta + \epsilon = \beta_0 + \beta_1X + \epsilon$$
		      where $\beta_0= 0.5, \quad \beta_1= -0.75, \quad X \sim (0.5, 2)$, and $\epsilon \sim N(0,1)$.Present
		      summary statistics
		\item Implement the Newton-Raphson algorithm for the MLE problem.
		      \begin{enumerate}
			      \item Report the estimated values for the three parameters.
			      \item Compute the Hessina at the estimated values. How is this connected
			            to the estimators' variance covariance MLE matrix?
		      \end{enumerate}
	\end{enumerate}
\end{homeworkProblem}
\end{document}
