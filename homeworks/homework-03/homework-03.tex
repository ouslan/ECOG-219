\documentclass{article}

\usepackage{fancyhdr}
\usepackage{extramarks}
\usepackage{amsmath}
\usepackage{amsthm}
\usepackage{amsfonts}
\usepackage{tikz}
\usepackage[plain]{algorithm}
\usepackage{algpseudocode}
\usepackage{pgfplots}
\pgfplotsset{compat=1.18}
\usetikzlibrary{automata,positioning}

%
% Basic Document Settings
%

\topmargin=-0.45in
\evensidemargin=0in
\oddsidemargin=0in
\textwidth=6.5in
\textheight=9.0in
\headsep=0.25in

\linespread{1.1}

\pagestyle{fancy}
\lhead{\hmwkAuthorName}
\chead{\hmwkClass\ (\hmwkClassInstructor): \hmwkTitle}
\rhead{\firstxmark}
\lfoot{\lastxmark}
\cfoot{\thepage}

\renewcommand\headrulewidth{0.4pt}
\renewcommand\footrulewidth{0.4pt}

\setlength\parindent{0pt}

%
% Create Problem Sections
%

\newcommand{\enterProblemHeader}[1]{
	\nobreak\extramarks{}{Problem \arabic{#1} continued on next page\ldots}\nobreak{}
	\nobreak\extramarks{Problem \arabic{#1} (continued)}{Problem \arabic{#1} continued on next page\ldots}\nobreak{}
}

\newcommand{\exitProblemHeader}[1]{
	\nobreak\extramarks{Problem \arabic{#1} (continued)}{Problem \arabic{#1} continued on next page\ldots}\nobreak{}
	\stepcounter{#1}
	\nobreak\extramarks{Problem \arabic{#1}}{}\nobreak{}
}

\setcounter{secnumdepth}{0}
\newcounter{partCounter}
\newcounter{homeworkProblemCounter}
\setcounter{homeworkProblemCounter}{1}
\nobreak\extramarks{Problem \arabic{homeworkProblemCounter}}{}\nobreak{}

%
% Homework Problem Environment
%
% This environment takes an optional argument. When given, it will adjust the
% problem counter. This is useful for when the problems given for your
% assignment aren't sequential. See the last 3 problems of this template for an
% example.
%
\newenvironment{homeworkProblem}[1][-1]{
	\ifnum#1>0
		\setcounter{homeworkProblemCounter}{#1}
	\fi
	\section{Problem \arabic{homeworkProblemCounter}}
	\setcounter{partCounter}{1}
	\enterProblemHeader{homeworkProblemCounter}
}{
	\exitProblemHeader{homeworkProblemCounter}
}

%
% Homework Details
%   - Title
%   - Due date
%   - Class
%   - Section/Time
%   - Instructor
%   - Author
%

\newcommand{\hmwkTitle}{Problem Set\ \#3}
\newcommand{\hmwkDueDate}{Jun 12, 2025}
\newcommand{\hmwkClass}{ECON 219}
\newcommand{\hmwkClassInstructor}{Dr. Sergio Urzua}
\newcommand{\hmwkAuthorName}{\textbf{Alejandro Ouslan}}

%
% Title Page
%

\title{
	\vspace{2in}
	\textmd{\textbf{\hmwkClass:\ \hmwkTitle}}\\
	\normalsize\vspace{0.1in}\small{Due\ on\ \hmwkDueDate}\\
	\vspace{0.1in}\large{\textit{\hmwkClassInstructor}}
	\vspace{3in}
}

\author{\hmwkAuthorName}
\date{}

\renewcommand{\part}[1]{\textbf{\large Part \Alph{partCounter}}\stepcounter{partCounter}\\}

%
% Various Helper Commands
%

% Useful for algorithms
\newcommand{\alg}[1]{\textsc{\bfseries \footnotesize #1}}

% For derivatives
\newcommand{\deriv}[1]{\frac{\mathrm{d}}{\mathrm{d}x} (#1)}

% For partial derivatives
\newcommand{\pderiv}[2]{\frac{\partial}{\partial #1} (#2)}

% Integral dx
\newcommand{\dx}{\mathrm{d}x}

% Alias for the Solution section header
\newcommand{\solution}{\textbf{\large Solution}}

% Probability commands: Expectation, Variance, Covariance, Bias
\newcommand{\E}{\mathrm{E}}
\newcommand{\Var}{\mathrm{Var}}
\newcommand{\Cov}{\mathrm{Cov}}
\newcommand{\Bias}{\mathrm{Bias}}

\begin{document}

\maketitle

\pagebreak

% Problem 1
\begin{homeworkProblem}
  Suppose that a certain wine dealer is in possesion of a particular quantity of wine, which he can either sell at the present time 
  $(i=0)$ for a sum of $K$ dollars of else store for some length of time and then sell at a higher value.
  \begin{enumerate}
    \item Present an expression for the growing value $(V)$ of the wine as a function of time. 
      
  \[
  V(t) = K e^{rt}
  \]

  where:
  \begin{itemize}
    \item \( K \): initial value of the wine at time \( t = 0 \),
    \item \( r \): appreciation rate,
    \item \( t \): time stored,
    \item \( V(t) \): value after time \( t \).
  \end{itemize}
    \item Explain under what assumptions the maximization of profits is the same as maxing the sales renueves (V)

 Let \( \rho \) be the discount rate (opportunity cost of capital). The present value of selling at time \( t \) is:

  \[
  \text{PV}(t) = V(t) e^{-\rho t} = K e^{(r - \rho)t}
  \]

  Maximizing profit is equivalent to maximizing \( V(t) \) only when \( \rho = 0 \), i.e., there is no discounting.
    \item Under the assumption of part b, present the maximization problem of the wine dealer. 
        Assuming storage cost \( C(t) \), the profit is:

  \[
  \Pi(t) = V(t) - C(t) = K e^{rt} - C(t)
  \]

  The wine dealer solves:

  \[
  \max_{t \geq 0} \Pi(t) = K e^{rt} - C(t)
  \]

    \item Present and intercept the first order condition of the preious problem. 
       Take derivative of \( \Pi(t) \):

  \[
  \frac{d\Pi}{dt} = \frac{dV}{dt} - \frac{dC}{dt} = K r e^{rt} - C'(t)
  \]

  FOC:

  \[
  K r e^{rt^*} = C'(t^*)
  \]

  \textit{Interpretation:} At the optimum time \( t^* \), marginal benefit equals marginal cost.

    \item Present and interpret the second order condition of the wine dealer's problem. 
       The second derivative:

  \[
  \frac{d^2\Pi}{dt^2} = K r^2 e^{rt} - C''(t)
  \]

  SOC for a maximum:

  \[
  K r^2 e^{rt^*} < C''(t^*)
  \]

  \textit{Interpretation:} Marginal cost increases faster than marginal benefit at \( t^* \).

    \item Obtain the optimum length of storage time. 
       From the FOC:

  \[
  K r e^{rt^*} = C'(t^*)
  \]

  Example: If \( C(t) = ct \), then \( C'(t) = c \). Solve:

  \[
  K r e^{rt^*} = c \Rightarrow e^{rt^*} = \frac{c}{K r} \Rightarrow t^* = \frac{1}{r} \ln\left( \frac{c}{K r} \right)
  \]
    \item Confirm that the previous answer defines a maximum and not a minimum. 
       With \( C(t) = c t \), \( C''(t) = 0 \). So second derivative is:

  \[
  \frac{d^2\Pi}{dt^2} = K r^2 e^{rt} > 0
  \]

  In this case, we get a minimum unless a more realistic cost function like \( C(t) = ct + dt^2 \) is used, where:

  \[
  C'(t) = c + 2dt, \quad C''(t) = 2d
  \]

  Then:

  \[
  \text{SOC: } K r^2 e^{rt^*} < 2d \quad \text{(ensures maximum)}
  \]
  \end{enumerate}
\end{homeworkProblem}

\begin{homeworkProblem}
  Consider the following firm's renueves function:
  $$R_1 = P_{10} Q_1 + P_{20}Q_2 $$
  where $Q_s$ represent the output level of the two different products the firm produces. The firm's cost is 
  assumed to be: 
  $$C =Q_1^2 + Q_1Q_1 + 2 Q_2^2$$
  \begin{enumerate}
    \item Present a plot of the marginal cost of the firm with refract to its first productita .
      Interpret you results. 
      \[
MC_1 = \frac{\partial C}{\partial Q_1} = \frac{\partial}{\partial Q_1} (2Q_1^2 + 2Q_2^2) = 4Q_1
\]

The marginal cost increases linearly with the quantity of the first good. This indicates increasing marginal cost or decreasing returns to scale for \( Q_1 \).
    \item Present the profit function of the firm. 
      \[
\pi(Q_1, Q_2) = P_{10} Q_1 + P_{20} Q_2 - (2Q_1^2 + 2Q_2^2)
\]
    \item Present the firm order condition of the firm's problem. Solve for the optimal level of the production. 
      \begin{align*}
\frac{\partial \pi}{\partial Q_1} &= P_{10} - 4Q_1 = 0 \quad \Rightarrow \quad Q_1^* = \frac{P_{10}}{4} \\
\frac{\partial \pi}{\partial Q_2} &= P_{20} - 4Q_2 = 0 \quad \Rightarrow \quad Q_2^* = \frac{P_{20}}{4}
\end{align*}
    \item Depict the supply function of both goods.
      \[
Q_1(P_{10}) = \frac{P_{10}}{4}, \quad Q_2(P_{20}) = \frac{P_{20}}{4}
\]
    \item Check the second order condition of the optimization problem. 
      \[
\frac{\partial^2 \pi}{\partial Q_1^2} = -4, \quad \frac{\partial^2 \pi}{\partial Q_2^2} = -4, \quad \frac{\partial^2 \pi}{\partial Q_1 \partial Q_2} = 0
\]
\[
H = \begin{bmatrix}
-4 & 0 \\
0 & -4
\end{bmatrix}
\]
    \item $c$ and $b$ using total derivatives. 
      \[
\frac{dQ_1}{dP_{10}} = \frac{1}{4}, \quad \frac{dQ_2}{dP_{20}} = \frac{1}{4}
\]
  \end{enumerate}
\end{homeworkProblem}

\end{document}

