\documentclass{article}

\usepackage{fancyhdr}
\usepackage{extramarks}
\usepackage{amsmath}
\usepackage{amsthm}
\usepackage{amsfonts}
\usepackage{tikz}
\usepackage[plain]{algorithm}
\usepackage{algpseudocode}

\usetikzlibrary{automata,positioning}

%
% Basic Document Settings
%

\topmargin=-0.45in
\evensidemargin=0in
\oddsidemargin=0in
\textwidth=6.5in
\textheight=9.0in
\headsep=0.25in

\linespread{1.1}

\pagestyle{fancy}
\lhead{\hmwkAuthorName}
\chead{\hmwkClass\ (\hmwkClassInstructor): \hmwkTitle}
\rhead{\firstxmark}
\lfoot{\lastxmark}
\cfoot{\thepage}

\renewcommand\headrulewidth{0.4pt}
\renewcommand\footrulewidth{0.4pt}

\setlength\parindent{0pt}

%
% Create Problem Sections
%

\newcommand{\enterProblemHeader}[1]{
	\nobreak\extramarks{}{Problem \arabic{#1} continued on next page\ldots}\nobreak{}
	\nobreak\extramarks{Problem \arabic{#1} (continued)}{Problem \arabic{#1} continued on next page\ldots}\nobreak{}
}

\newcommand{\exitProblemHeader}[1]{
	\nobreak\extramarks{Problem \arabic{#1} (continued)}{Problem \arabic{#1} continued on next page\ldots}\nobreak{}
	\stepcounter{#1}
	\nobreak\extramarks{Problem \arabic{#1}}{}\nobreak{}
}

\setcounter{secnumdepth}{0}
\newcounter{partCounter}
\newcounter{homeworkProblemCounter}
\setcounter{homeworkProblemCounter}{1}
\nobreak\extramarks{Problem \arabic{homeworkProblemCounter}}{}\nobreak{}

%
% Homework Problem Environment
%
% This environment takes an optional argument. When given, it will adjust the
% problem counter. This is useful for when the problems given for your
% assignment aren't sequential. See the last 3 problems of this template for an
% example.
%
\newenvironment{homeworkProblem}[1][-1]{
	\ifnum#1>0
		\setcounter{homeworkProblemCounter}{#1}
	\fi
	\section{Problem \arabic{homeworkProblemCounter}}
	\setcounter{partCounter}{1}
	\enterProblemHeader{homeworkProblemCounter}
}{
	\exitProblemHeader{homeworkProblemCounter}
}

%
% Homework Details
%   - Title
%   - Due date
%   - Class
%   - Section/Time
%   - Instructor
%   - Author
%

\newcommand{\hmwkTitle}{Problem Set\ \#1}
\newcommand{\hmwkDueDate}{May 29, 2025}
\newcommand{\hmwkClass}{ECOG 219}
\newcommand{\hmwkClassInstructor}{Dr. Sergio Urzua}
\newcommand{\hmwkAuthorName}{\textbf{Alejandro Ouslan}}

%
% Title Page
%

\title{
	\vspace{2in}
	\textmd{\textbf{\hmwkClass:\ \hmwkTitle}}\\
	\normalsize\vspace{0.1in}\small{Due\ on\ \hmwkDueDate}\\
	\vspace{0.1in}\large{\textit{\hmwkClassInstructor}}
	\vspace{3in}
}

\author{\hmwkAuthorName}
\date{}

\renewcommand{\part}[1]{\textbf{\large Part \Alph{partCounter}}\stepcounter{partCounter}\\}

%
% Various Helper Commands
%

% Useful for algorithms
\newcommand{\alg}[1]{\textsc{\bfseries \footnotesize #1}}

% For derivatives
\newcommand{\deriv}[1]{\frac{\mathrm{d}}{\mathrm{d}x} (#1)}

% For partial derivatives
\newcommand{\pderiv}[2]{\frac{\partial}{\partial #1} (#2)}

% Integral dx
\newcommand{\dx}{\mathrm{d}x}

% Alias for the Solution section header
\newcommand{\solution}{\textbf{\large Solution}}

% Probability commands: Expectation, Variance, Covariance, Bias
\newcommand{\E}{\mathrm{E}}
\newcommand{\Var}{\mathrm{Var}}
\newcommand{\Cov}{\mathrm{Cov}}
\newcommand{\Bias}{\mathrm{Bias}}

\begin{document}

\maketitle

\pagebreak

% Homework problem 2.4.5
\begin{homeworkProblem}
  If the domain of the function $y= 5+ =x$ is the set $\{x | 1 \ge x \ge 9\}$, find the range 
  of the function and express it as a set.
  \[
    \begin{split}
      DOM = \{x | 1 \ge x \ge 9\} \\
      RANGE = \{x | 32 \ge x \ge 8\}
    \end{split}
  \]  
\end{homeworkProblem}

% Homework problem 2.4.7
\begin{homeworkProblem}
  In the theory of the firm, economists consider the total cost $C$ to be a function of 
  the output level $Q:C= f(Q)$.
  \begin{enumerate}
    \item According to the definition of a function, should each cost figure 
      be associated with a unique level of output? \\
      \textbf{Answer:} Not nevesary since there could be multiple outputs $(Q)$ that could 
      be map to the same $C$ but each $Q$ can only map to one $C$.
    \item Should each level of output determine a unique cost figure? \\ 
      \textbf{Answer:} Yes, given that each $Q$ can only map to one $C$.

  \end{enumerate}
\end{homeworkProblem}

% Homework problem 2.4.8
\begin{homeworkProblem}
   If an output level $Q_1$ can be produced at a cost of $C_1$ , then it must also be possible (by
being less effcient) to produce $Q_1$ at a cost of $C_1 + \$1$, or  $C_1 + \$2$, and so on. Thus it
would seem that output Q does not uniquely determine total cost $C$. If so, to write
$C = f (Q)$ would violate the deffnition of a function. How, in spite of the this reasoning,
would you justify the use of the function $C = f (Q)$?

\textbf{Answer:} The function would still be valid since adding some cost constant $\forall a\in \mathbb{R}^+$ 
would only displace the cost function upwards by $a$ units. A cost increas of $a$ units could be described 
in the following method:
\[
  \begin{split}
    C = f(Q) \\
    C + a = f(Q) + a ; \forall a \in \mathbb{R}^+
  \end{split}
\]
\end{homeworkProblem}

% Homework problem 2.5.7
\begin{homeworkProblem}
  Show that $X^{m/n}= \sqrt[n][x^m] = (\sqrt[n][x])^m$.Spevify the rules applied in each 
  step.
  \[
\begin{split}
  x^{m/n} &= x^{m \frac{1}{n}} \\
          &= (x^m)^{\frac{1}{n}} \\
          &= \sqrt[n][x^m]
\end{split}
  \]
  \[
\begin{split}
  x^{m/n} &= x^{m \frac{1}{n}} \\
          &=  x^{\frac{1}{n} m} \\
          &= (x^{\frac{1}{n}})^m \\
          &= (\sqrt[n][x])^m
\end{split}
  \]
\end{homeworkProblem}

% Homework problem 2.5.8
\begin{homeworkProblem}
Prove Rule VI and Rule VII:
		      \begin{enumerate}
			      \item \textbf{Rule VI:} $(x^m)^n=x^{m \cdot n}$
			            \begin{proof}
				            To prove that $(x^m)^n=x^{m \cdot n}$ is true we use the
				            induction method. First, we prove that:
				            \[
					            P_1: (x^m)^1 \Rightarrow x^m = x^m
				            \]
				            \[
					            P_2: (x^m)^2 \Rightarrow x^{m} \cdot x^{m} \Rightarrow x^{m+m} = x^{m \cdot 2}
				            \]
				            \[
					            P_n: (x^m)^n \Rightarrow x^{m}_1 \cdot x^{m}_2 \cdot ... \cdot x^{m}_{n} \Rightarrow x^{m+m+...+m} = x^{m \cdot n}
				            \]
				            Then we prove that $P_1$ hold for all $P_{n+1}$, that is:
				            \[
					            P_{n+1}: (x^m)^n \Rightarrow x^{m}_1 \cdot x^{m}_2 \cdot ... \cdot x^{m}_{n} \cdot x^{m}_{n+1}
				            \]
				            \[
					            \Rightarrow x^{m \cdot n} \cdot x^{m}
				            \]
				            \[
					            \Rightarrow x^{m \dot n + m} = x^{m \cdot (n+1)}
				            \]
				            Therefore $P_{n+1}$ holds for all positive integers n since $P_1 \implies P_{n+1}$.
			            \end{proof}
			      \item \textbf{Rule VII:} $x^m \cdot y^m = (x \cdot y)^m$
			            \begin{proof}
				            To prove that $x^m \cdot y^m = (x \cdot y)^m$ is true we use the
				            induction method. First, we prove that:
				            \[
					            P_1: x^1 \cdot y^1 \Rightarrow x \cdot y = x \cdot y
				            \]
				            \[
					            P_2: x^2 \cdot y^2 \Rightarrow x \cdot y \cdot x \cdot y \Rightarrow (x \cdot y)^2 = (x \cdot y)^2
				            \]
				            \[
					            P_n: x^m \cdot y^m \Rightarrow x_1 \cdot y_1 \cdot x_2 \cdot y_2 \cdot ... \cdot x_n \cdot y_n \Rightarrow (x \cdot y)^m
				            \]
				            Then we prove that $P_1$ hold for all $P_{n+1}$, that is:
				            \[
					            P_{n+1}: x^{m+1} \cdot y^{m+1} \Rightarrow x_1 \cdot y_1 \cdot x_2 \cdot y_2 \cdot ... \cdot x_n \cdot y_n \cdot x_{m+1} \cdot y_{m+1}
				            \]
				            \[
					            \Rightarrow (x \cdot y)^{m} \cdot (x \cdot y)_{m+1}
				            \]
				            \[
					            \Rightarrow (x \cdot y)^{m+1}
				            \]
				            Therefore $P_{n+1}$ holds for all positive integers n since $P_1 \implies P_{n+1}$.
			            \end{proof}
		      \end{enumerate}

\end{homeworkProblem}

% Homework problem 3.2.1
\begin{homeworkProblem}
  Given the market model
  \[
  \begin{split}
    Q_d &= Q_s \\
    Q_d &= 21 - 3P \\ 
    Q_s &= -4 + 8P \\
  \end{split}
\]
Find $P^*$ and $P^*$ by elimination of variables.
\[
\begin{split}
  Q_d &= Q_s \\
    Q_d &= 21 - 3P \\ 
    Q_s &= -4 + 8P \\
    21 - 3P &= 3P + 8P \\
    25 &= 11P \\
    P^*&= \frac{25}{11} \\
    Q^*&= \frac{156}{11} \\
\end{split}
\]

Using the formaulas:
\[
\begin{split}
  P^* &= \frac{a + c}{b + d} \\
  &= \frac{21 + 4}{3 + 8} \\
  &= \frac{25}{11} \\ 
  Q^* &= \frac{ad - bc}{b + d} \\
      &= \frac{21 \cdot 8 - 3\cdot 4}{3 + 8} \\
      &= \frac{156}{11}
\end{split}
\]
\end{homeworkProblem}

% Homework problem 3.3.6
\begin{homeworkProblem}
  Find the equilibrion solution for each of the following models:
  \begin{enumerate}
    \item 
  \[
  \begin{split}
    Q_d &= Q_s \\
    Q_d &= 3 - P^2 \\ 
    Q_s &= 6P - 4 \\
  \end{split}
\]
  \textbf{Answer:}
  \[
  \begin{split}
    Q_d &= Q_s \\
    Q_d &= 3 - P^2 \\ 
    Q_s &= 6P - 4 \\
    3 -p^2 &= 6p-4 \\
    p^2 6p + 7 &= 0 \\
    (p+7)(p-1) &= 0 \\ 
    P&=1 \\
    Q*&=2
  \end{split}
\]

\item
  \[
  \begin{split}
    Q_d &= Q_s \\
    Q_d &= 8 - P^2 \\ 
    Q_s &= P^2 -2 \\
    P^2-2 &= 8 - p^2  \\
    2(P^2 -5) &= 0 \\ 
    P^* &= \sqrt{5} \\
    Q*  &= 3
  \end{split}
\]
  \end{enumerate}
\end{homeworkProblem}

% Homework problem 3.3.7
\begin{homeworkProblem}
  The market equilibrium condition, $Q_d = Q_s$, is often expressed in an equivalent 
  alternative from, $Q_d - Q_s = 0$, which has the economiv interpretation "excess demand 
  is zero". Does (3.7) represent this latter version of the equilibrium condition? if not, 
  supply an appropriate economic interpretation for (3.7).
  \[
\begin{split}
  Q_d &= Q_s \\
  Q_d - Q_s &= 0\\ 
  4 - p^2 - 4p + 1 &= 0 \\ 
  p^2 + 4p - 5 &= 0
\end{split}
  \]
\end{homeworkProblem}

% Homework problem 3.4.3
\begin{homeworkProblem}
The demand and supply function of a two-commodity market model are as follows: 
\[
\begin{split}
  Q_{d_1} &= 18 - 3 P_1 + P_2 \\ 
  Q_{d_2} &= 12 + P_1 - 2 P_2 \\ 
  Q_{s_1} &= -2 + 4P_1 \\ 
  Q_{s_2} &= -2 + 3 P_2 \\
  -7p_1 +  P_2 &= -20 \\
  P_1 + 5 P_2 &= -14 \\ 
  \begin{bmatrix}
    -7 & 1 \\
    1 & -5 \\
    \end{bmatrix} \begin{bmatrix}
  P_1 \\ 
  P_2 \\
    \end{bmatrix} = \begin{bmatrix}
  -20 \\ -14
  \end{bmatrix} \\
\left[\begin{array}{cc|c}
-7 & 1 & -20 \\
1 & -5 & -14 \\
\end{array}\right] \\ 
\left[\begin{array}{cc|c}
1 & -5 & -14 \\
0 & 1 & \frac{59}{17} \\
\end{array}\right] \\ 
\left[\begin{array}{cc|c}
1 & 0 & \frac{57}{17} \\
0 & 1 & \frac{59}{19} \\
\end{array}\right]
\end{split}
\]
\end{homeworkProblem}

% Homework problem 2.5.2
\begin{homeworkProblem}
   Let the national-income model be:
		      \[
			      \begin{split}
				      Y &= C + I_0 + G \\
				      C &= a + b(Y - T_0) \\
				      G &= gY \\
			      \end{split}
		      \]
		      \begin{enumerate}
			      \item Identify the endogenous variables.

			            \textbf{Answer:} $Y,C,G$ are endogenous.
			      \item Give the economic meaning of the parameter g.

			            \textbf{Answer:}
			      \item Find the equilibrium national income.

			            \textbf{Answer:}
			      \item What restriction on the parameters is needed for a solution to exist?

			            \textbf{Answer:}
		      \end{enumerate}
\end{homeworkProblem}

% Homework problem 4.1.2
\begin{homeworkProblem}
  Rewrite the market model (3.12) in the format of (4.1) with the variables arranged in
            the following order: $Q_{d1}$ , $Q_{s1}$ , $Q_{d2}$ , $Q_{s2}$ , $P_1$ , $P_2$ . Write out the coefficient matrix, the
            variable vector, and the constant vector.
            \[
                \begin{bmatrix}
                    1 & -1 & 0 & 0 & 0 & 0 \\
                    1 & 0 & 0 & 0 & -a_1 & -a_2 \\
                    0 & 1 & 0 & 0 & -b_1 & -b_2 \\
                    0 & 0 & 1 & -1 & 0 & 0 \\
                    0 & 0 & 1 & 0 & -\alpha_1 & -\alpha_2 \\
                    0 & 0 & 0 & 1 & \beta_1 & -\beta_2 \\ 
                \end{bmatrix}
                \begin{bmatrix}
                    Q_{d1} \\
                    Q_{s1} \\
                    Q_{d2} \\
                    Q_{s2} \\
                    P_1 \\
                    P_2
                \end{bmatrix}
                \begin{bmatrix}
                    0 \\
                    a_0 \\
                    b_0 \\
                    0 \\
                    \alpha_1 \\
                    \beta_1 \\
                \end{bmatrix}
            \]

\end{homeworkProblem}

% Homework problem 4.1.3
\begin{homeworkProblem}
\end{homeworkProblem}

% Homework problem 4.3.3
\begin{homeworkProblem}
\end{homeworkProblem}

% Homework problem 4.3.8
\begin{homeworkProblem}
\end{homeworkProblem}

% Homework problem 4.4.8
\begin{homeworkProblem}
\end{homeworkProblem}

% Homework problem 4.5.4
\begin{homeworkProblem}
\end{homeworkProblem}

% Homework problem 4.6.1
\begin{homeworkProblem}
\end{homeworkProblem}

% Homework problem 4.6.2
\begin{homeworkProblem}
\end{homeworkProblem}

% Homework problem 4.6.6
\begin{homeworkProblem}
\end{homeworkProblem}

% Homework problem 4.7.1
\begin{homeworkProblem}
\end{homeworkProblem}

\end{document}
