\documentclass{article}

\usepackage{fancyhdr}
\usepackage{extramarks}
\usepackage{amsmath}
\usepackage{amsthm}
\usepackage{amsfonts}
\usepackage{tikz}
\usepackage[plain]{algorithm}
\usepackage{algpseudocode}

\usetikzlibrary{automata,positioning}

%
% Basic Document Settings
%

\topmargin=-0.45in
\evensidemargin=0in
\oddsidemargin=0in
\textwidth=6.5in
\textheight=9.0in
\headsep=0.25in

\linespread{1.1}

\pagestyle{fancy}
\lhead{\hmwkAuthorName}
\chead{\hmwkClass\ (\hmwkClassInstructor): \hmwkTitle}
\rhead{\firstxmark}
\lfoot{\lastxmark}
\cfoot{\thepage}

\renewcommand\headrulewidth{0.4pt}
\renewcommand\footrulewidth{0.4pt}

\setlength\parindent{0pt}

%
% Create Problem Sections
%

\newcommand{\enterProblemHeader}[1]{
	\nobreak\extramarks{}{Problem \arabic{#1} continued on next page\ldots}\nobreak{}
	\nobreak\extramarks{Problem \arabic{#1} (continued)}{Problem \arabic{#1} continued on next page\ldots}\nobreak{}
}

\newcommand{\exitProblemHeader}[1]{
	\nobreak\extramarks{Problem \arabic{#1} (continued)}{Problem \arabic{#1} continued on next page\ldots}\nobreak{}
	\stepcounter{#1}
	\nobreak\extramarks{Problem \arabic{#1}}{}\nobreak{}
}

\setcounter{secnumdepth}{0}
\newcounter{partCounter}
\newcounter{homeworkProblemCounter}
\setcounter{homeworkProblemCounter}{1}
\nobreak\extramarks{Problem \arabic{homeworkProblemCounter}}{}\nobreak{}

%
% Homework Problem Environment
%
% This environment takes an optional argument. When given, it will adjust the
% problem counter. This is useful for when the problems given for your
% assignment aren't sequential. See the last 3 problems of this template for an
% example.
%
\newenvironment{homeworkProblem}[1][-1]{
	\ifnum#1>0
		\setcounter{homeworkProblemCounter}{#1}
	\fi
	\section{Problem \arabic{homeworkProblemCounter}}
	\setcounter{partCounter}{1}
	\enterProblemHeader{homeworkProblemCounter}
}{
	\exitProblemHeader{homeworkProblemCounter}
}

%
% Homework Details
%   - Title
%   - Due date
%   - Class
%   - Section/Time
%   - Instructor
%   - Author
%

\newcommand{\hmwkTitle}{Problem Set\ \#2}
\newcommand{\hmwkDueDate}{May 29, 2025}
\newcommand{\hmwkClass}{ECON 219}
\newcommand{\hmwkClassInstructor}{Dr. Sergio Urzua}
\newcommand{\hmwkAuthorName}{\textbf{Alejandro Ouslan}}

%
% Title Page
%

\title{
	\vspace{2in}
	\textmd{\textbf{\hmwkClass:\ \hmwkTitle}}\\
	\normalsize\vspace{0.1in}\small{Due\ on\ \hmwkDueDate}\\
	\vspace{0.1in}\large{\textit{\hmwkClassInstructor}}
	\vspace{3in}
}

\author{\hmwkAuthorName}
\date{}

\renewcommand{\part}[1]{\textbf{\large Part \Alph{partCounter}}\stepcounter{partCounter}\\}

%
% Various Helper Commands
%

% Useful for algorithms
\newcommand{\alg}[1]{\textsc{\bfseries \footnotesize #1}}

% For derivatives
\newcommand{\deriv}[1]{\frac{\mathrm{d}}{\mathrm{d}x} (#1)}

% For partial derivatives
\newcommand{\pderiv}[2]{\frac{\partial}{\partial #1} (#2)}

% Integral dx
\newcommand{\dx}{\mathrm{d}x}

% Alias for the Solution section header
\newcommand{\solution}{\textbf{\large Solution}}

% Probability commands: Expectation, Variance, Covariance, Bias
\newcommand{\E}{\mathrm{E}}
\newcommand{\Var}{\mathrm{Var}}
\newcommand{\Cov}{\mathrm{Cov}}
\newcommand{\Bias}{\mathrm{Bias}}

\begin{document}

\maketitle

\pagebreak

% Problem 1
\begin{homeworkProblem}
  Consider the National Income Model:
  \[
  \begin{split}
    Y &= C + I_0 + G_0 \\ 
    C &= \alpha + \beta (Y - T) \\
    T &= \gamma + \delta Y
  \end{split}
  \]
  \begin{enumerate}
    \item Define and interpret each of the componnts of the model. 
      Identify parameter and variables.
    \item Impose reasonable assumptions on the signs and values of the parameters.
    \item Solve for the equilibriom income. 
    \item Obtain an discuss the six comprarative-static derivatives.
  \end{enumerate}
\end{homeworkProblem}

% Problem 2
\begin{homeworkProblem}
  COnsider the market model:
  \[
    \begin{split}
    Q_s &= Q_s \\
    Q_s &= D(P,Y_0) \\
    Q_d &= S(P)
    \end{split}
  \]
  \begin{enumerate}
    \item Provide an economic interpretation to each of the equaions. In you answers, 
      include the assumptions on the signs of the relevant derivaties. 
    \item Define the concept of market equilibriom. Provide and explain its mathematical formulation.
    \item Show that:
      $$
      \frac{dP^*}{dY_0} > 0
      $$
    Where $P^*$ is the equilibrium price. Provide an economiv interpretation for this result.
    \item Show that:
      $$
      \frac{dQ^*}{dY_0} > 0
      $$
    Where $Q^*$ is the equilibrium price. Provide an economiv interpretation for this result.
  \item Answer quesitons c and d using total derivatives.


  \end{enumerate}
\end{homeworkProblem}


\end{document}
