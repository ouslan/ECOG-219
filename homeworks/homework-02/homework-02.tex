\documentclass{article}

\usepackage{fancyhdr}
\usepackage{extramarks}
\usepackage{amsmath}
\usepackage{amsthm}
\usepackage{amsfonts}
\usepackage{tikz}
\usepackage[plain]{algorithm}
\usepackage{algpseudocode}

\usetikzlibrary{automata,positioning}

%
% Basic Document Settings
%

\topmargin=-0.45in
\evensidemargin=0in
\oddsidemargin=0in
\textwidth=6.5in
\textheight=9.0in
\headsep=0.25in

\linespread{1.1}

\pagestyle{fancy}
\lhead{\hmwkAuthorName}
\chead{\hmwkClass\ (\hmwkClassInstructor): \hmwkTitle}
\rhead{\firstxmark}
\lfoot{\lastxmark}
\cfoot{\thepage}

\renewcommand\headrulewidth{0.4pt}
\renewcommand\footrulewidth{0.4pt}

\setlength\parindent{0pt}

%
% Create Problem Sections
%

\newcommand{\enterProblemHeader}[1]{
	\nobreak\extramarks{}{Problem \arabic{#1} continued on next page\ldots}\nobreak{}
	\nobreak\extramarks{Problem \arabic{#1} (continued)}{Problem \arabic{#1} continued on next page\ldots}\nobreak{}
}

\newcommand{\exitProblemHeader}[1]{
	\nobreak\extramarks{Problem \arabic{#1} (continued)}{Problem \arabic{#1} continued on next page\ldots}\nobreak{}
	\stepcounter{#1}
	\nobreak\extramarks{Problem \arabic{#1}}{}\nobreak{}
}

\setcounter{secnumdepth}{0}
\newcounter{partCounter}
\newcounter{homeworkProblemCounter}
\setcounter{homeworkProblemCounter}{1}
\nobreak\extramarks{Problem \arabic{homeworkProblemCounter}}{}\nobreak{}

%
% Homework Problem Environment
%
% This environment takes an optional argument. When given, it will adjust the
% problem counter. This is useful for when the problems given for your
% assignment aren't sequential. See the last 3 problems of this template for an
% example.
%
\newenvironment{homeworkProblem}[1][-1]{
	\ifnum#1>0
		\setcounter{homeworkProblemCounter}{#1}
	\fi
	\section{Problem \arabic{homeworkProblemCounter}}
	\setcounter{partCounter}{1}
	\enterProblemHeader{homeworkProblemCounter}
}{
	\exitProblemHeader{homeworkProblemCounter}
}

%
% Homework Details
%   - Title
%   - Due date
%   - Class
%   - Section/Time
%   - Instructor
%   - Author
%

\newcommand{\hmwkTitle}{Problem Set\ \#2}
\newcommand{\hmwkDueDate}{May 29, 2025}
\newcommand{\hmwkClass}{ECON 219}
\newcommand{\hmwkClassInstructor}{Dr. Sergio Urzua}
\newcommand{\hmwkAuthorName}{\textbf{Alejandro Ouslan}}

%
% Title Page
%

\title{
	\vspace{2in}
	\textmd{\textbf{\hmwkClass:\ \hmwkTitle}}\\
	\normalsize\vspace{0.1in}\small{Due\ on\ \hmwkDueDate}\\
	\vspace{0.1in}\large{\textit{\hmwkClassInstructor}}
	\vspace{3in}
}

\author{\hmwkAuthorName}
\date{}

\renewcommand{\part}[1]{\textbf{\large Part \Alph{partCounter}}\stepcounter{partCounter}\\}

%
% Various Helper Commands
%

% Useful for algorithms
\newcommand{\alg}[1]{\textsc{\bfseries \footnotesize #1}}

% For derivatives
\newcommand{\deriv}[1]{\frac{\mathrm{d}}{\mathrm{d}x} (#1)}

% For partial derivatives
\newcommand{\pderiv}[2]{\frac{\partial}{\partial #1} (#2)}

% Integral dx
\newcommand{\dx}{\mathrm{d}x}

% Alias for the Solution section header
\newcommand{\solution}{\textbf{\large Solution}}

% Probability commands: Expectation, Variance, Covariance, Bias
\newcommand{\E}{\mathrm{E}}
\newcommand{\Var}{\mathrm{Var}}
\newcommand{\Cov}{\mathrm{Cov}}
\newcommand{\Bias}{\mathrm{Bias}}

\begin{document}

\maketitle

\pagebreak

% Problem 1
\begin{homeworkProblem}
  Consider the National Income Model:
  \[
  \begin{split}
    Y &= C + I_0 + G_0 \\ 
    C &= \alpha + \beta (Y - T) \\
    T &= \gamma + \delta Y
  \end{split}
  \]
  \begin{enumerate}
    \item Define and interpret each of the components of the model. 
      Identify parameter and variables.
      \begin{enumerate}
        \item $Y = C + I_0 + G_0$ Nations income 
        \item $C = \alpha + \beta (Y - T)$ Consumtion function 
        \item $T = \gamma + \delta Y$ Tax Function 
        \item $I,G$ investments and goverment spending are exogenous
        \item $\alpha$ consumption intercept (minimum consumption when there is no income)
        \item $\beta$ Marginal propensity to consume 
        \item $\gamma$ minimum taxes 
        \item $\delta$ tax rate on income
      \end{enumerate}
    \item Impose reasonable assumptions on the signs and values of the parameters.
    \begin{enumerate}
      \item $\alpha > 0$ there is at least some consumption even if there is no income
      \item $0<\beta <1$ hoesholds consume at least something and not all of their income 
      \item $\gamma \ge 0$ there is no negitive taxes 
      \item $I_0 > 0$ There is some investments
      \item $G_0 > 0$ There is some goverment spending
    \end{enumerate}
    \item Solve for the equilibrium income. 
      Start from:
    \[
      \begin{split}
        Y &= C + I_0 + G_0 \\
          &= \alpha + \beta (Y - T) + I_0 + G_0 \\
          &= \alpha + \beta(Y - \gamma - \delta Y) + I_0 + G_0 \\
          &= \alpha + \beta Y - \beta \gamma - \beta \delta Y + I_0 + G_0 \\
          &= Y(1 - \beta + \beta \delta) = \alpha - \beta \gamma + I_0 + G_0 \\
          &= \frac{\alpha - \beta \gamma + I_0 + G_0}{1 - \beta(1 - \delta)}
      \end{split}
    \]
    \item Obtain an discuss the six comparative-static derivatives.
      Let:
\[
Y = \frac{\alpha - \beta \gamma + I_0 + G_0}{1 - \beta(1 - \delta)}
\]

\textbf{Partial derivatives:}

\begin{align*}
\frac{\partial Y}{\partial \alpha} &= \frac{1}{1 - \beta(1 - \delta)} > 0 \\
\frac{\partial Y}{\partial \gamma} &= \frac{-\beta}{1 - \beta(1 - \delta)} < 0 \\
\frac{\partial Y}{\partial I_0} &= \frac{1}{1 - \beta(1 - \delta)} > 0 \\
\frac{\partial Y}{\partial G_0} &= \frac{1}{1 - \beta(1 - \delta)} > 0 \\
\frac{\partial Y}{\partial \delta} &= \frac{\beta(\alpha - \beta \gamma + I_0 + G_0)}{[1 - \beta(1 - \delta)]^2} > 0 \quad \text{(if numerator positive)} \\
\frac{\partial Y}{\partial \beta} &= \frac{ -\gamma(1 - \beta(1 - \delta)) + (\alpha - \beta \gamma + I_0 + G_0)(1 - \delta) }{[1 - \beta(1 - \delta)]^2} \quad \text{(ambiguous)}
\end{align*}

\textbf{Summary of signs:}

\begin{center}
\begin{tabular}{|c|c|c|}
\hline
\textbf{Parameter} & \textbf{Partial Derivative} & \textbf{Sign} \\
\hline
$\alpha$ & $\frac{\partial Y}{\partial \alpha}$ & Positive \\
$\beta$  & $\frac{\partial Y}{\partial \beta}$ & Ambiguous \\
$\gamma$ & $\frac{\partial Y}{\partial \gamma}$ & Negative \\
$\delta$ & $\frac{\partial Y}{\partial \delta}$ & Positive (if numerator $> 0$) \\
$I_0$    & $\frac{\partial Y}{\partial I_0}$    & Positive \\
$G_0$    & $\frac{\partial Y}{\partial G_0}$    & Positive \\
\hline
\end{tabular}
\end{center}
  \end{enumerate}
\end{homeworkProblem}

% Problem 2
\begin{homeworkProblem}
  Consider the market model:
  \[
    \begin{split}
    Q_s &= Q_d \\
    Q_d &= D(P,Y_0) \\
    Q_s &= S(P)
    \end{split}
  \]
  \begin{enumerate}
    \item Provide an economic interpretation to each of the equations. In you answers, 
      include the assumptions on the signs of the relevant derivatives. 
      \begin{enumerate}
        \item $Q_s = Q_d$ the quantity supplied is equal to the quantity produced ($Q_s,Q_s \in \mathbb{N}$)
        \item $Q_d = D(P,Y_0)$ the quantity depanded is a function of the price and income ($P,Y_0,D(\cdot) \in \mathbb{R}^+$)
        \item $Q_s = S(P)$ the quantity supplied is a function of the price ($P,S(\cdot) \in \mathbb{R}^+$)
      \end{enumerate}
    \item Define the concept of market equilibrium. Provide and explain its mathematical formulation.
      
      \textbf{Answer:} The market equilibruim is when the $P^*$ is picked such that $D(P^*,Y_0) = S(P^*)$. $Q^* = D(P^*,Y_0) = S(P^*)$ is 
      the equilibruim quantity produced
    \item Show that:
      $$
      \frac{dP^*}{dY_0} > 0
      $$
    Where $P^*$ is the equilibrium price. Provide an economic interpretation for this result.
    \item Show that:
      $$
      \frac{dQ^*}{dY_0} > 0
      $$
    Where $Q*$ is the equilibrium price. Provide an economic interpretation for this result.
    \[
      \begin{split}
        \frac{dS}{dP} \cdot \frac{dP^*}{dY_0} = \frac{dP^*}{dY_0}  > 0 \\
      \end{split}
    \]
    \textbf{Answer:} An increase in income leads to higher demand, which pushes up both the price and the equilibrium quantity. Suppliers respond to higher prices by increasing supply, leading to a higher $Q^*$
  \item Answer questions c and d using total derivatives.
    \[
      \begin{split}
        \frac{dS}{dP} \cdot \frac{dP^*}{dY_0} &= \frac{\partial D}{\partial P} \cdot \frac{dP^*}{dY_0} + \frac{\partial D}{\partial Y_0} \\
        \frac{dP^*}{dY_0} &= \frac{\frac{\partial D}{\partial Y_0}}{\frac{dS}{dP} - \frac{\partial D}{\partial P}} > 0 \\
        \frac{dQ^*}{dY_0} &= \frac{dS}{dP} \cdot \frac{dP^*}{dY_0} > 0
      \end{split}
    \]
  \end{enumerate}
\end{homeworkProblem}


\end{document}
