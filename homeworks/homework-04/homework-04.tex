\documentclass{article}

\usepackage{fancyhdr}
\usepackage{extramarks}
\usepackage{amsmath}
\usepackage{amsthm}
\usepackage{amsfonts}
\usepackage{tikz}
\usepackage[plain]{algorithm}
\usepackage{algpseudocode}
\usepackage{pgfplots}
\pgfplotsset{compat=1.18}
\usetikzlibrary{automata,positioning}

%
% Basic Document Settings
%

\topmargin=-0.45in
\evensidemargin=0in
\oddsidemargin=0in
\textwidth=6.5in
\textheight=9.0in
\headsep=0.25in

\linespread{1.1}

\pagestyle{fancy}
\lhead{\hmwkAuthorName}
\chead{\hmwkClass\ (\hmwkClassInstructor): \hmwkTitle}
\rhead{\firstxmark}
\lfoot{\lastxmark}
\cfoot{\thepage}

\renewcommand\headrulewidth{0.4pt}
\renewcommand\footrulewidth{0.4pt}

\setlength\parindent{0pt}

%
% Create Problem Sections
%

\newcommand{\enterProblemHeader}[1]{
	\nobreak\extramarks{}{Problem \arabic{#1} continued on next page\ldots}\nobreak{}
	\nobreak\extramarks{Problem \arabic{#1} (continued)}{Problem \arabic{#1} continued on next page\ldots}\nobreak{}
}

\newcommand{\exitProblemHeader}[1]{
	\nobreak\extramarks{Problem \arabic{#1} (continued)}{Problem \arabic{#1} continued on next page\ldots}\nobreak{}
	\stepcounter{#1}
	\nobreak\extramarks{Problem \arabic{#1}}{}\nobreak{}
}

\setcounter{secnumdepth}{0}
\newcounter{partCounter}
\newcounter{homeworkProblemCounter}
\setcounter{homeworkProblemCounter}{1}
\nobreak\extramarks{Problem \arabic{homeworkProblemCounter}}{}\nobreak{}

%
% Homework Problem Environment
%
% This environment takes an optional argument. When given, it will adjust the
% problem counter. This is useful for when the problems given for your
% assignment aren't sequential. See the last 3 problems of this template for an
% example.
%
\newenvironment{homeworkProblem}[1][-1]{
	\ifnum#1>0
		\setcounter{homeworkProblemCounter}{#1}
	\fi
	\section{Problem \arabic{homeworkProblemCounter}}
	\setcounter{partCounter}{1}
	\enterProblemHeader{homeworkProblemCounter}
}{
	\exitProblemHeader{homeworkProblemCounter}
}

%
% Homework Details
%   - Title
%   - Due date
%   - Class
%   - Section/Time
%   - Instructor
%   - Author
%

\newcommand{\hmwkTitle}{Problem Set\ \#4}
\newcommand{\hmwkDueDate}{Jun 12, 2025}
\newcommand{\hmwkClass}{ECON 219}
\newcommand{\hmwkClassInstructor}{Dr. Sergio Urzua}
\newcommand{\hmwkAuthorName}{\textbf{Alejandro Ouslan}}

%
% Title Page
%

\title{
	\vspace{2in}
	\textmd{\textbf{\hmwkClass:\ \hmwkTitle}}\\
	\normalsize\vspace{0.1in}\small{Due\ on\ \hmwkDueDate}\\
	\vspace{0.1in}\large{\textit{\hmwkClassInstructor}}
	\vspace{3in}
}

\author{\hmwkAuthorName}
\date{}

\renewcommand{\part}[1]{\textbf{\large Part \Alph{partCounter}}\stepcounter{partCounter}\\}

%
% Various Helper Commands
%

% Useful for algorithms
\newcommand{\alg}[1]{\textsc{\bfseries \footnotesize #1}}

% For derivatives
\newcommand{\deriv}[1]{\frac{\mathrm{d}}{\mathrm{d}x} (#1)}

% For partial derivatives
\newcommand{\pderiv}[2]{\frac{\partial}{\partial #1} (#2)}

% Integral dx
\newcommand{\dx}{\mathrm{d}x}

% Alias for the Solution section header
\newcommand{\solution}{\textbf{\large Solution}}

% Probability commands: Expectation, Variance, Covariance, Bias
\newcommand{\E}{\mathrm{E}}
\newcommand{\Var}{\mathrm{Var}}
\newcommand{\Cov}{\mathrm{Cov}}
\newcommand{\Bias}{\mathrm{Bias}}

\begin{document}

\maketitle

\pagebreak

% Problem 1
\begin{homeworkProblem}
	Consider a monopolistic firm operating in three differenc markets. Its revenue and cost function:
	\[
		\begin{split}
			R=R_1(Q_1) + R_2(Q_2) +R_3(Q_3) \\
			C = C(Q) \text{ Where } Q = Q_1 + Q_2 + Q_3.
		\end{split}
	\]
	\begin{enumerate}
		\item Define the profit maximization problem of the firm.
		      \[
			      \Pi = R(Q_1, Q_2, Q_3) - C(Q_1 + Q_2 + Q_3)
		      \]

		      Substituting the expressions for revenue and cost:

		      \[
			      \Pi = \left( R_1(Q_1) + R_2(Q_2) + R_3(Q_3) \right) - C(Q_1 + Q_2 + Q_3)
		      \]

		      Thus, the firm's profit maximization problem is:

		      \[
			      \max_{Q_1, Q_2, Q_3} \, \left( R_1(Q_1) + R_2(Q_2) + R_3(Q_3) - C(Q_1 + Q_2 + Q_3) \right)
		      \]
		\item Present the first-order condition (set of equations).
		      \[
			      \frac{\partial \Pi}{\partial Q_1} = \frac{dR_1(Q_1)}{dQ_1} - \frac{dC(Q)}{dQ_1} = 0
		      \]

		      \[
			      \frac{\partial \Pi}{\partial Q_2} = \frac{dR_2(Q_2)}{dQ_2} - \frac{dC(Q)}{dQ_2} = 0
		      \]

		      \[
			      \frac{\partial \Pi}{\partial Q_3} = \frac{dR_3(Q_3)}{dQ_3} - \frac{dC(Q)}{dQ_3} = 0
		      \]

		\item Provide an economic interpretation to the firs order condition. Specifically,
		      connect marginal revenues and demand elasticities to explain under what condition the frim
		      will change a higher price.
		      \[
			      MR_1(Q_1) = MC(Q)
		      \]
		      \[
			      MR_2(Q_2) = MC(Q)
		      \]
		      \[
			      MR_3(Q_3) = MC(Q)
		      \]

		      Where:
		      - \(MR_i(Q_i) = \frac{dR_i}{dQ_i}\) is the marginal revenue in market \(i\).
		      - \(MC(Q) = \frac{dC}{dQ}\) is the marginal cost of production, where \(Q = Q_1 + Q_2 + Q_3\).

		      This implys that the firm will set the marginal revenue equal to marginal cost for each market.

		\item Present the Hessian of the firm's objective function.
		      \[
			      \Pi(Q_1, Q_2, Q_3) = R_1(Q_1) + R_2(Q_2) + R_3(Q_3) - C(Q_1 + Q_2 + Q_3)
		      \]

		      The Hessian matrix \(H\) is given by:

		      \[
			      H =
			      \begin{bmatrix}
				      \frac{\partial^2 \Pi}{\partial Q_1^2}            & \frac{\partial^2 \Pi}{\partial Q_1 \partial Q_2} & \frac{\partial^2 \Pi}{\partial Q_1 \partial Q_3} \\
				      \frac{\partial^2 \Pi}{\partial Q_2 \partial Q_1} & \frac{\partial^2 \Pi}{\partial Q_2^2}            & \frac{\partial^2 \Pi}{\partial Q_2 \partial Q_3} \\
				      \frac{\partial^2 \Pi}{\partial Q_3 \partial Q_1} & \frac{\partial^2 \Pi}{\partial Q_3 \partial Q_2} & \frac{\partial^2 \Pi}{\partial Q_3^2}
			      \end{bmatrix}
		      \]

		      Where the second derivatives are:

		      \[
			      \frac{\partial^2 \Pi}{\partial Q_i^2} = \frac{d^2 R_i(Q_i)}{dQ_i^2} - \frac{d^2 C(Q)}{dQ_i^2}
		      \]

		      \[
			      \frac{\partial^2 \Pi}{\partial Q_i \partial Q_j} = \frac{d^2 R_i(Q_i)}{dQ_i dQ_j} - \frac{d^2 C(Q)}{dQ_i dQ_j}
		      \]
		\item Assume each of the revenue function is concave and convex cost. Would this structure
		      secure the second- order condition? Explain.
		      \begin{enumerate}
			      \item \textbf{Concave Revenue Functions:} If the revenue functions \(R_1(Q_1)\), \(R_2(Q_2)\), and \(R_3(Q_3)\)
			            are concave, their second derivatives will be non-positive. This implies that the marginal revenue functions
			            are decreasing, which is necessary for profit maximization.
			      \item \textbf{Convex Cost Function:} If the cost function \(C(Q)\) is convex, then its second derivative will
			            be non-decreasing (positive), meaning the marginal cost increases with total output.
		      \end{enumerate}

	\end{enumerate}
\end{homeworkProblem}

%problem 2 
\begin{homeworkProblem}
	Let $U = U(x,y)$ be the utility function of the agent, $x$ and $y$ repesents the good. Assume positive
	marginal utilities. Let $P_X$, $P_y$ be the associated prices and $I$ income.
	\begin{enumerate}
		\item State the agent's utility maximization problem.
		      \[
			      \max_{x, y} U(x, y)
		      \]
		      subject to the budget constraint:
		      \[
			      P_x x + P_y y = I
		      \]
		      where \( x, y \geq 0 \) and \( I, P_x, P_y > 0 \).
		\item Present the Langrangian funciton.
		      \[
			      \mathcal{L}(x, y, \lambda) = U(x, y) + \lambda \left( I - P_x x - P_y y \right)
		      \]
		\item Derive and interpret the first order condition. In your analysis, you must include the Langrge multiplier.
		      \[
			      \frac{\partial \mathcal{L}}{\partial x} = \frac{\partial U(x, y)}{\partial x} - \lambda P_x = 0
		      \]
		      \[
			      \frac{\partial U(x, y)}{\partial x} = \lambda P_x
		      \]

		      \[
			      \frac{\partial \mathcal{L}}{\partial y} = \frac{\partial U(x, y)}{\partial y} - \lambda P_y = 0
		      \]
		      \[
			      \frac{\partial U(x, y)}{\partial y} = \lambda P_y
		      \]
		      \[
			      \frac{\partial \mathcal{L}}{\partial \lambda} = I - P_x x - P_y y = 0
		      \]
		      \[
			      P_x x + P_y y = I \quad
		      \]
		\item Present a graphical interpretation of the optimality condition (include indifferent curves and budget sets).

		      \[
			      P_x x + P_y y = I
		      \]
		      \[
			      \text{MRS} = \frac{\partial U(x, y) / \partial x}{\partial U(x, y) / \partial y} = \frac{P_x}{P_y}
		      \]

		      This is the optimality condition.

		\item Present the second- order condition and descrive the condition under which one could secure a maximum.
		\item Solve the previous questions but assuming:
		      \begin{enumerate}
			      \item $U(x,y) = x^a y^b$ where $a + b <1 $
			            \[
				            \mathcal{L}(x, y, \lambda) = x^a y^b + \lambda (I - P_x x - P_y y)
			            \]
			            \[
				            a x^{a-1} y^b = \lambda P_x \quad \text{(FOC for \( x \))}
			            \]
			            \[
				            b x^a y^{b-1} = \lambda P_y \quad \text{(FOC for \( y \))}
			            \]
			            \[
				            \frac{a}{b} = \frac{P_x}{P_y}
			            \]
			      \item $U(x,y) = ax + by$
			            \[
				            \mathcal{L}(x, y, \lambda) = ax + by + \lambda (I - P_x x - P_y y)
			            \]
			            \[
				            a = \lambda P_x \quad \text{(FOC for \( x \))}
			            \]
			            \[
				            b = \lambda P_y \quad \text{(FOC for \( y \))}
			            \]
			            \[
				            \lambda = \frac{a}{P_x} = \frac{b}{P_y}
			            \]
			      \item $U(x,y) = Min(x,y)$
			            \[
				            \mathcal{L}(x, y, \lambda) = \min(x, y) + \lambda (I - P_x x - P_y y)
			            \]
			            \[
				            P_x x + P_y x = I
			            \]


		      \end{enumerate}
	\end{enumerate}

\end{homeworkProblem}
\end{document}
