\documentclass{article}

\usepackage{fancyhdr}
\usepackage{extramarks}
\usepackage{amsmath}
\usepackage{amsthm}
\usepackage{amsfonts}
\usepackage{tikz}
\usepackage[plain]{algorithm}
\usepackage{algpseudocode}
\usepackage{pgfplots}
\pgfplotsset{compat=1.18}
\usetikzlibrary{automata,positioning}

%
% Basic Document Settings
%

\topmargin=-0.45in
\evensidemargin=0in
\oddsidemargin=0in
\textwidth=6.5in
\textheight=9.0in
\headsep=0.25in

\linespread{1.1}

\pagestyle{fancy}
\lhead{\hmwkAuthorName}
\chead{\hmwkClass\ (\hmwkClassInstructor): \hmwkTitle}
\rhead{\firstxmark}
\lfoot{\lastxmark}
\cfoot{\thepage}

\renewcommand\headrulewidth{0.4pt}
\renewcommand\footrulewidth{0.4pt}

\setlength\parindent{0pt}

%
% Create Problem Sections
%

\newcommand{\enterProblemHeader}[1]{
	\nobreak\extramarks{}{Problem \arabic{#1} continued on next page\ldots}\nobreak{}
	\nobreak\extramarks{Problem \arabic{#1} (continued)}{Problem \arabic{#1} continued on next page\ldots}\nobreak{}
}

\newcommand{\exitProblemHeader}[1]{
	\nobreak\extramarks{Problem \arabic{#1} (continued)}{Problem \arabic{#1} continued on next page\ldots}\nobreak{}
	\stepcounter{#1}
	\nobreak\extramarks{Problem \arabic{#1}}{}\nobreak{}
}

\setcounter{secnumdepth}{0}
\newcounter{partCounter}
\newcounter{homeworkProblemCounter}
\setcounter{homeworkProblemCounter}{1}
\nobreak\extramarks{Problem \arabic{homeworkProblemCounter}}{}\nobreak{}

%
% Homework Problem Environment
%
% This environment takes an optional argument. When given, it will adjust the
% problem counter. This is useful for when the problems given for your
% assignment aren't sequential. See the last 3 problems of this template for an
% example.
%
\newenvironment{homeworkProblem}[1][-1]{
	\ifnum#1>0
		\setcounter{homeworkProblemCounter}{#1}
	\fi
	\section{Problem \arabic{homeworkProblemCounter}}
	\setcounter{partCounter}{1}
	\enterProblemHeader{homeworkProblemCounter}
}{
	\exitProblemHeader{homeworkProblemCounter}
}

%
% Homework Details
%   - Title
%   - Due date
%   - Class
%   - Section/Time
%   - Instructor
%   - Author
%

\newcommand{\hmwkTitle}{Problem Set\ \#4}
\newcommand{\hmwkDueDate}{Jun 12, 2025}
\newcommand{\hmwkClass}{ECON 219}
\newcommand{\hmwkClassInstructor}{Dr. Sergio Urzua}
\newcommand{\hmwkAuthorName}{\textbf{Alejandro Ouslan}}

%
% Title Page
%

\title{
	\vspace{2in}
	\textmd{\textbf{\hmwkClass:\ \hmwkTitle}}\\
	\normalsize\vspace{0.1in}\small{Due\ on\ \hmwkDueDate}\\
	\vspace{0.1in}\large{\textit{\hmwkClassInstructor}}
	\vspace{3in}
}

\author{\hmwkAuthorName}
\date{}

\renewcommand{\part}[1]{\textbf{\large Part \Alph{partCounter}}\stepcounter{partCounter}\\}

%
% Various Helper Commands
%

% Useful for algorithms
\newcommand{\alg}[1]{\textsc{\bfseries \footnotesize #1}}

% For derivatives
\newcommand{\deriv}[1]{\frac{\mathrm{d}}{\mathrm{d}x} (#1)}

% For partial derivatives
\newcommand{\pderiv}[2]{\frac{\partial}{\partial #1} (#2)}

% Integral dx
\newcommand{\dx}{\mathrm{d}x}

% Alias for the Solution section header
\newcommand{\solution}{\textbf{\large Solution}}

% Probability commands: Expectation, Variance, Covariance, Bias
\newcommand{\E}{\mathrm{E}}
\newcommand{\Var}{\mathrm{Var}}
\newcommand{\Cov}{\mathrm{Cov}}
\newcommand{\Bias}{\mathrm{Bias}}

\begin{document}

\maketitle

\pagebreak

% Problem 1
\begin{homeworkProblem}
	Consider a monopolistic firm operating in three different markets. Its revenue and cost function:
	\[
		\begin{split}
			R=R_1(Q_1) + R_2(Q_2) +R_3(Q_3) \\
			C = C(Q) \text{ Where } Q = Q_1 + Q_2 + Q_3.
		\end{split}
	\]
	\begin{enumerate}
		\item Define the profit maximization problem of the firm.
		\item Present the first-order condition (set of equations).
		\item Provide an economic interpretation to the firs order condition. Specifically,
		      connect marginal revenues and demand elasticities to explain under what condition the frim
		      will change a higher price.
		\item Present the Hessian of the firm's objective function.
		\item Assume each of the revenue function is concave and convex cost. Would this structure
		      secure the second- order condition? Explain.
	\end{enumerate}
\end{homeworkProblem}

%problem 1 
Let $U = U(x,y)$ be the utility function of the agent, $x$ and $y$ repesents the good. Assume positive
marginal utilities. Let $P_X$, $P_y$ be the associated prices and $I$ income.
\begin{enumerate}
	\item State the agent's utility maximization problem.
	\item Present the Langrangian funciton.
	\item Derive and interpret the first order condition. In your analysis, you must include the Langrge multiplier.
	\item Present a graphical interpretation of the optimality condition (include indifferent curves and budget sets).
	\item Present the second- order condition and descrive the condition under which one could secure a maximum.
	\item Solve the previous questions but assuming:
	      \begin{enumerate}
		      \item $U(x,y) = x^a y^b$ where $a + b <1 $
		      \item $U(x,y) = ax + by$
		      \item $U(x,y) = Min(x,y)$
	      \end{enumerate}
\end{enumerate}
\end{document}
